\documentclass{bbawslides}

\usepackage{ifthen}

\usepackage[a4paper]{hyperref}
\rotateheaderstrue	%%-- otherwise seminar does strange things
\usepackage{url}

\usepackage[ngerman]{babel}
\usepackage[babel]{csquotes}
\usepackage[autoplay]{animate}
\usepackage{graphicx}


\begin{document}
\providecommand{\Title}{}


\begin{bbawtitle}[(Open-Source-)OCR-Workflows]
  \vspace*{3em}%
  Kay-Michael Würzner\\[-.25em]%
  \textcolor{urlColor}{\texttt{{\small wuerzner@bbaw.de}}}
  \\[3em]
  {\footnotesize{%
    DH-Kolloquium an der Berlin-Brandenburgischen Akademie der Wissenschaften\\%
    4. August 2017\\%
  }}
\end{bbawtitle}
\slideStyleFrame

\renewcommand{\footerText}{\tiny 4. August 2017, DH-Kolloquium, BBAW}

%----------------------------------------------------------------------------------------------
% Outline
%----------------------------------------------------------------------------------------------
\begin{bbawslide}{Übersicht}
  \vspace*{7mm}%
  \centerslidestrue%
  \begin{itemize}
  \item Was ist OCR?
  \item Wozu braucht man OCR?
  \item Komponenten eines einfachen OCR-Workflows
  \item Modelltraining
  \item Optimierungsoptionen
  \item Komplexere OCR-Workflows
  \item Warum braucht man OCR (i.e. DH-Perspektiven)?
  \item OCR-D
  \end{itemize}
\end{bbawslide}

\begin{bbawpart}{\Large\bf Was ist OCR?}
\end{bbawpart}

\begin{bbawslide}{Was ist OCR?}
  \vspace*{7mm}%
  \centerslidestrue%
  \begin{tabular}{lc}
    \begin{minipage}{0.6\textwidth}
      \begin{itemize}
        \item \textbf{O}ptical \textbf{C}haracter \textbf{R}ecognition: Automatische Erfassung von Text in Bildern
        \item ursprünglich begrenzt auf Zeichenerkennung
        \item heute häufig Synonym für den gesamten Texterfassungsprozess
        \begin{itemize}
          \item Bildvorverarbeitung
          \item Layoutanalyse (OLR)
          \item Zeilenerkennung
          \item \ldots
        \end{itemize}
      \end{itemize}
    \end{minipage}
    &
    \begin{minipage}{0.4\textwidth}
      \epsfig{file=figures/times.eps,width=\textwidth}
    \end{minipage}
  \end{tabular}
\end{bbawslide}

\begin{bbawslide}{Zeichenorientierte Ansätze}
  \vspace*{7mm}%
  \hspace*{-2em}%
  \centerslidestrue%
  \begin{tabular}{lc}
    \begin{minipage}{0.68\textwidth}
      \begin{mitemize}
      \item Erkennung erfolgt \emph{glyphenweise}
        \begin{description}\small
          \item[Pattern matching:] Vergleich der Zeichenbilder zu in einem \enquote{Setzkasten} gespeicherten Glyphen \textbf{Pixel für Pixel}
          \item[Feature extraction:] Zerlegung der Glyphen in vordefinierte, bedeutungstragende \textbf{Eigenschaften}
          wie \emph{Einfärbung, Kurven, Linien} etc. und Vergleich zu Trainingsmaterialien
        \end{description}
      \item Kombination beider Ansätze möglich
      \item Zerlegung der Seite in \emph{Zeilen} und \emph{Zeichen} notwendig
      \item Open-Source Software \texttt{Tesseract 3} \hlcite{Smith 2007}
        \begin{mitemize}\small
          \item Einsatz verschiedener Lexika für frequente Wörter, Sonderzeichen und
        häufige Fehler zur Verbesserung der Erkennung
        \end{mitemize}
      \end{mitemize}
    \end{minipage}
    &
    \begin{minipage}{0.38\textwidth}
      \epsfig{file=figures/char.eps,width=\textwidth}
    \end{minipage}
  \end{tabular}
\end{bbawslide}

\begin{bbawslide}{Sequenzorientierte Ansätze}
  \vspace*{7mm}%
  \centerslidestrue%
  \begin{minipage}{1.05\textwidth}
    \begin{itemize}
      \item Erkennung erfolgt \emph{zeilenweise}
        \begin{description}\small
          \item[Scaling:] Einheitliche Höhe für alle Zeilen
          \item[Feature extraction:] Grid mit festgelegter Anzahl (horizontaler) Zeilen und variabler Anzahl
                                     (vertikaler) Spalten: Zeilen als Sequenzen binärwertiger Vektoren fixer Länge
        \end{description}
    \end{itemize}
  \end{minipage}
  \begin{center}
    \epsfig{file=figures/grid.eps,width=1.05\textwidth}
  \end{center}
  \begin{minipage}{1.05\textwidth}
    \begin{itemize}
      \item Kontextsensitive (i.e.~ über \emph{Übergangswahrscheinlichkeiten} der Vektoren) Erkennung
      \item Zerlegung der Seite in \emph{Zeilen} notwendig
      \item Vorgehen (normalerweise) \emph{robuster} gegenüber Varianz durch Artefakte als zeichenorientierte Ansätze
      \item Open-Source Software \texttt{OCRopus} \hlcite{Breuel 2008}
      \begin{mitemize}\small
        \item Einsatz \emph{neuronaler Netzwerke} für die Sequenzklassifikation
      \end{mitemize}
    \end{itemize}
  \end{minipage}
\end{bbawslide}

\begin{bbawpart}{\Large\bf Wozu braucht man OCR?}
\end{bbawpart}

\renewcommand{\footerText}{\tiny 4. August 2017, DH-Kolloquium, BBAW\hspace{4cm} Image by Achim Raschka, CC BY-SA 3.0}

\begin{bbawslide}{Wozu braucht man OCR?}
  \vspace*{1mm}%
  \centerslidesfalse%
  \begin{tabular}{cc}
    \raisebox{-\height}{\parbox{7cm}{%
      \begin{itemize}
        \item typische Anwendungen:
        \begin{itemize}\small
          \item Nummernschilderkennung (\emph{Automatic number plate recognition})
        \end{itemize}
      \end{itemize}
    }}
    &
    \raisebox{-\height}{\epsfig{file=figures/ANPR.eps,width=0.4\textwidth}}%
  \end{tabular}
\end{bbawslide}

\renewcommand{\footerText}{\tiny 4. August 2017, DH-Kolloquium, BBAW\hspace{4cm} Image by JD, CC BY-SA 2.0}

\begin{bbawslide}{Wozu braucht man OCR?}
  \vspace*{1mm}%
  \centerslidesfalse%
  \begin{tabular}{cc}
    \raisebox{-\height}{\parbox{7cm}{%
      \begin{itemize}
        \item typische Anwendungen:
        \begin{itemize}\small
          \item Nummernschilderkennung (\emph{Automatic number plate recognition})
          \item Captcha-Umgehung (\emph{Completely Automated Public Turing test to tell Computers and Humans Apart})
        \end{itemize}
      \end{itemize}
    }}
    &
    \raisebox{-\height}{\epsfig{file=figures/Captcha.eps,width=0.4\textwidth}}%
  \end{tabular}
\end{bbawslide}

\renewcommand{\footerText}{\tiny 4. August 2017, DH-Kolloquium, BBAW\hspace{4cm} Image by Eluminary, CC BY-SA 4.0}

\begin{bbawslide}{Wozu braucht man OCR?}
  \vspace*{1mm}%
  \centerslidesfalse%
  \begin{tabular}{cc}
    \raisebox{-\height}{\parbox{7cm}{%
      \begin{itemize}
        \item typische Anwendungen:
        \begin{itemize}\small
          \item Nummernschilderkennung (\emph{Automatic number plate recognition})
          \item Captcha-Umgehung (\emph{Completely Automated Public Turing test to tell Computers and Humans Apart})
          \item Schlüsselinformationsextraktion (\emph{Document's key information extraction})
        \end{itemize}
      \end{itemize}
    }}
    &
    \raisebox{-\height}{\epsfig{file=figures/deposit.eps,width=0.4\textwidth}}%
  \end{tabular}
\end{bbawslide}

\renewcommand{\footerText}{\tiny 4. August 2017, DH-Kolloquium, BBAW}

\begin{bbawslide}{Wozu braucht man OCR?}
  \vspace*{1mm}%
  \centerslidesfalse%
  \begin{tabular}{cc}
    \raisebox{-\height}{\parbox{7cm}{%
      \begin{itemize}
        \item typische Anwendungen:
        \begin{itemize}\small
          \item Nummernschilderkennung (\emph{Automatic number plate recognition})
          \item Captcha-Umgehung (\emph{Completely Automated Public Turing test to tell Computers and Humans Apart})
          \item Schlüsselinformationsextraktion (\emph{Document's key information extraction})
          \item Handschrifterkennung (\emph{Handwritten text recognition})
        \end{itemize}
      \end{itemize}
    }}
    &
    \raisebox{-\height}{\epsfig{file=figures/transkribus.eps,width=0.4\textwidth}}%
  \end{tabular}
\end{bbawslide}

\renewcommand{\footerText}{\tiny 4. August 2017, DH-Kolloquium, BBAW\hspace{4cm} Images by Uwe Springmann, CC BY-SA 4.0}

\begin{bbawslide}{Wozu braucht man OCR?}
  \vspace*{1mm}%
  \centerslidesfalse%
  \begin{tabular}{cc}
    \raisebox{-\height}{\parbox{7cm}{%
      \begin{itemize}
        \item typische Anwendungen:
        \begin{itemize}\small
          \item Nummernschilderkennung (\emph{Automatic number plate recognition})
          \item Captcha-Umgehung (\emph{Completely Automated Public Turing test to tell Computers and Humans Apart})
          \item Schlüsselinformationsextraktion (\emph{Document's key information extraction})
          \item Handschrifterkennung (\emph{Handwritten text recognition})
          \item \textbf{Volltextdigitalisierung}
        \end{itemize}
      \end{itemize}
    }}
    &
    \raisebox{-\height}{\epsfig{file=figures/beauvais_0.eps,width=0.4\textwidth}}%
  \end{tabular}
\end{bbawslide}

\begin{bbawslide}{Wozu braucht man OCR?}
  \vspace*{1mm}%
  \centerslidesfalse%
  \begin{tabular}{cc}
    \raisebox{-\height}{\parbox{7cm}{%
      \begin{itemize}
        \item typische Anwendungen:
        \begin{itemize}\small
          \item Nummernschilderkennung (\emph{Automatic number plate recognition})
          \item Captcha-Umgehung (\emph{Completely Automated Public Turing test to tell Computers and Humans Apart})
          \item Schlüsselinformationsextraktion (\emph{Document's key information extraction})
          \item Handschrifterkennung (\emph{Handwritten text recognition})
          \item \textbf{Volltextdigitalisierung}
        \end{itemize}
      \end{itemize}
    }}
    &
    \raisebox{-\height}{\epsfig{file=figures/beauvais_1.eps,width=0.4\textwidth}}%
  \end{tabular}
\end{bbawslide}

\renewcommand{\footerText}{\tiny 4. August 2017, DH-Kolloquium, BBAW}

\begin{bbawpart}{\Large\bf Komponenten eines OCR-Workflows}
\end{bbawpart}

\begin{bbawslide}{Komponenten eines OCR-Workflows}
  \vspace*{2mm}%
  \centerslidestrue%
  \begin{tabular}{cc}
    \phantom{\raisebox{-3\height}{\parbox{7cm}{%
      \begin{enumerate}
        \item Bildvorverarbeitung
        \item Layoutanalyse
        \item Texterkennung
      \end{enumerate}
    }}}
    &
    \raisebox{-\height}{\epsfig{file=figures/grenzboten_raw.eps,width=0.4\textwidth}}%
  \end{tabular}
\end{bbawslide}

\begin{bbawslide}{Komponenten eines OCR-Workflows}
  \vspace*{2mm}%
  \centerslidestrue%
  \begin{tabular}{cc}
    \raisebox{-3\height}{\parbox{7cm}{%
      \begin{enumerate}
        \item Bildvorverarbeitung
        \item
        \item
      \end{enumerate}
    }}
    &
    \raisebox{-\height}{\epsfig{file=figures/grenzboten_raw.eps,width=0.4\textwidth}}%
  \end{tabular}
\end{bbawslide}

\begin{bbawslide}{Komponenten eines OCR-Workflows}
  \vspace*{2mm}%
  \centerslidestrue%
  \begin{tabular}{cc}
    \raisebox{-3\height}{\parbox{7cm}{%
      \begin{enumerate}
        \item Bildvorverarbeitung
        \item
        \item
      \end{enumerate}
    }}
    &
    \raisebox{-\height}{\epsfig{file=figures/grenzboten_opt.eps,width=0.4\textwidth}}%
  \end{tabular}
\end{bbawslide}

\begin{bbawslide}{Komponenten eines OCR-Workflows}
  \vspace*{2mm}%
  \centerslidestrue%
  \begin{tabular}{cc}
    \raisebox{-3\height}{\parbox{7cm}{%
      \begin{enumerate}
        \item Bildvorverarbeitung
        \item Layoutanalyse
        \item
      \end{enumerate}
    }}
    &
    \raisebox{-\height}{\epsfig{file=figures/grenzboten_opt.eps,width=0.4\textwidth}}%
  \end{tabular}
\end{bbawslide}

\begin{bbawslide}{Komponenten eines OCR-Workflows}
  \vspace*{2mm}%
  \centerslidestrue%
  \begin{tabular}{cc}
    \raisebox{-3\height}{\parbox{7cm}{%
      \begin{enumerate}
        \item Bildvorverarbeitung
        \item Layoutanalyse
        \item
      \end{enumerate}
    }}
    &
    \raisebox{-\height}{\epsfig{file=figures/grenzboten_struct.eps,width=0.4\textwidth}}%
  \end{tabular}
\end{bbawslide}

\begin{bbawslide}{Komponenten eines OCR-Workflows}
  \vspace*{2mm}%
  \centerslidestrue%
  \begin{tabular}{cc}
    \raisebox{-3.001\height}{\parbox{7cm}{%
      \begin{enumerate}
        \item Bildvorverarbeitung
        \item Layoutanalyse
        \item Texterkennung
      \end{enumerate}
    }}
    &
    \raisebox{-\height}{\epsfig{file=figures/grenzboten_struct.eps,width=0.4\textwidth}}%
  \end{tabular}
\end{bbawslide}

\begin{bbawslide}{Komponenten eines OCR-Workflows}
  \vspace*{2mm}%
  \centerslidestrue%
  \begin{tabular}{cc}
    \raisebox{-3.001\height}{\parbox{7cm}{%
      \begin{enumerate}
        \item Bildvorverarbeitung
        \item Layoutanalyse
        \item Texterkennung
      \end{enumerate}
    }}
    &
    \raisebox{-\height}{\epsfig{file=figures/grenzboten_text.eps,width=0.4\textwidth}}%
  \end{tabular}
\end{bbawslide}

\begin{bbawslide}{Komponenten eines OCR-Workflows: Bildvorverarbeitung}
  \vspace*{7mm}%
  \centerslidestrue%
  \begin{itemize}
    \item Prozesse zur bestmöglichen Vorbereitung der Digitalisate für OLR und OCR
    \begin{itemize}\small
      \item \textbf{Cropping:} Beschneidung des Digitalisats auf den Druckbereich
      \item \textbf{Deskewing:} Rotation des Digitalisats zur Begradigung von Schrägstellungen
      \item \textbf{Binarization:} Binäre Kodierung der Pixel (bedruckte Bereiche schwarz, nicht-bedruckte Bereiche weiß)
      \item \textbf{Despeckling:} Entfernung von Bildartefakten (Verschmutzungen, sichtbare Papiermaserung etc.)
      \item \textbf{Dewarping:} Begradigung von Wellen auf Zeilenebene
    \end{itemize}
    \item starker Einfluss auf die Erkennungsqualität
    \item besondere Relevanz für historische Vorlagen
  \end{itemize}
\end{bbawslide}

\begin{bbawslide}{Komponenten eines OCR-Workflows: Cropping}
  \vspace*{2mm}%
  \begin{center}
    \begin{tabular}{c@{\hspace{1cm}}c}
      \raisebox{-\height}{\epsfig{file=figures/cropping_in.eps,width=0.3\textwidth}}%
      &
      \raisebox{-\height}{\epsfig{file=figures/cropping_out.eps,width=0.3\textwidth}}%
    \end{tabular}
  \end{center}
\end{bbawslide}

\begin{bbawslide}{Komponenten eines OCR-Workflows: Deskewing}
  \vspace*{2mm}%
  \begin{center}
    \begin{tabular}{c@{\hspace{1cm}}c}
      \raisebox{-\height}{\epsfig{file=figures/deskewing_in.eps,width=0.3\textwidth}}%
      &
      \raisebox{-\height}{\epsfig{file=figures/deskewing_out.eps,width=0.3\textwidth}}%
    \end{tabular}
  \end{center}
\end{bbawslide}

\begin{bbawslide}{Komponenten eines OCR-Workflows: Binarization}
  \vspace*{2mm}%
  \begin{center}
    \begin{tabular}{c@{\hspace{1cm}}c}
      \raisebox{-\height}{\epsfig{file=figures/binarization_in.eps,width=0.4\textwidth}}%
      &
      \raisebox{-\height}{\epsfig{file=figures/binarization_out.eps,width=0.4\textwidth}}%
    \end{tabular}
  \end{center}
\end{bbawslide}

\begin{bbawslide}{Komponenten eines OCR-Workflows: Despeckling}
  \vspace*{2mm}%
  \begin{center}
    \begin{tabular}{c@{\hspace{1cm}}c}
      \raisebox{-\height}{\epsfig{file=figures/despeckling_in.eps,width=0.4\textwidth}}%
      &
      \raisebox{-\height}{\epsfig{file=figures/despeckling_out.eps,width=0.4\textwidth}}%
    \end{tabular}
  \end{center}
\end{bbawslide}

\begin{bbawslide}{Komponenten eines OCR-Workflows: Dewarping}
  \vspace*{2mm}%
  \begin{center}
      \epsfig{file=figures/dewarping_in_out.eps,width=0.9\textwidth}%
  \end{center}
\end{bbawslide}

\begin{bbawslide}{Komponenten eines OCR-Workflows: Bildvorverarbeitung}
  \vspace*{7mm}%
  \centerslidestrue%
  \textbf{Werkzeuge:}
  \begin{itemize}
    \item Bestandteil der meisten OCR-Programme, häufig jedoch nicht modular
    \item spezielle Tools
    \begin{itemize}\small
      \item \textbf{Scantailor} \url{https://github.com/scantailor/scantailor}
      \begin{description}\footnotesize
        \item[+] umfassendes, frei verfügbares Werkzeug
        \item[-] keine Programmierschnittstelle (API)
      \end{description}
      \item \textbf{Olena/SCRIBO} \url{https://www.lrde.epita.fr/wiki/Olena/Modules#SCRIBO}
      \begin{description}\footnotesize
        \item[+] frei verfügbare Programmierbibliothek für Deskewing, Binarisierung (Implementierung verschiedener Ansätze)
        \item[-] keine Weiterentwicklung/Pflege, schlechtes API-Design
      \end{description}
      \item \textbf{Unpaper} \url{https://github.com/Flameeyes/unpaper}
      \begin{description}\footnotesize
        \item[+] frei verfügbare Programmierbibliothek für Deskewing und Despeckling
      \end{description}
    \end{itemize}
  \end{itemize}
\end{bbawslide}

\begin{bbawslide}{Komponenten eines OCR-Workflows: Bildvorverarbeitung}
  \vspace*{7mm}%
  \centerslidestrue%
  \textbf{Werkzeuge:}
  \begin{itemize}
    \item teilweise auch in Bildbearbeitungsbibliotheken integriert
    \begin{itemize}\small
      \item \textbf{ImageMagick} \url{https://www.imagemagick.org/}
      \begin{description}\footnotesize
        \item[+] extrem umfangreiches, frei verfügbares Softwarepaket
        \item[-] keine spezifische OCR-Implementierung (aber: \url{http://www.fmwconcepts.com/imagemagick/})
      \end{description}
      \item \textbf{Leptonica} \url{http://www.leptonica.com/}
      \begin{description}\footnotesize
        \item[+] sehr umfangreiches, frei verfügbares Softwarepaket
        \item[+] Anwendung in \texttt{Tesseract}
      \end{description}
    \end{itemize}
    \item zahlreiche \textbf{wissenschaftliche Veröffentlichungen} zu einzelnen Aspekten
    \item \textbf{wissenschaftliche Wettbewerbe} zu ausgewählten Aspekten (insb. Binarization und Deskewing)
    \item Forschungsergebnisse finden \textbf{kaum Eingang in die Praxis}
  \end{itemize}
\end{bbawslide}

\begin{bbawslide}{Komponenten eines OCR-Workflows: Layoutanalyse}
  \vspace*{7mm}%
  \centerslidestrue%
  \begin{itemize}
    \item Prozesse zur Erkennung der Struktur auf Seiten- und Dokumentebene
    \begin{itemize}\small
      \item \textbf{Page Segmentation:} Lokalisierung von zusammenhängenden Text- und Nichttextbereichen
      \item \textbf{Region Classification:} Typisierung von Textbereichen
      \item \textbf{Line/Character Splitting:} Lokalisierung der einzelnen Zeilen/Zeichen
      \item \textbf{Document Analysis:} Konstruktion der logischen Dokumentstruktur (METS!)
    \end{itemize}
    \item entscheidend für die korrekte \textbf{Rekonstruktion des Textflusses} und damit für maschinelle Auswertungen
  \end{itemize}
\end{bbawslide}

\begin{bbawslide}{Komponenten eines OCR-Workflows: Layoutanalyse}
  \vspace*{2mm}%
  \centerslidesfalse%
  \begin{tabular}{cc}
    \raisebox{-\height}{\parbox{7cm}{%
      \begin{itemize}
        \item \textbf{strukturierende} Elemente
        \begin{itemize}
          \item Absätze
          \item Überschriften
        \end{itemize}
      \end{itemize}
    }}
    &
    \raisebox{-\height}{\epsfig{file=figures/layout_rec_example.eps,width=0.4\textwidth}}%
  \end{tabular}
\end{bbawslide}

\begin{bbawslide}{Komponenten eines OCR-Workflows: Layoutanalyse}
  \vspace*{2mm}%
  \centerslidesfalse%
  \begin{tabular}{cc}
    \raisebox{-\height}{\parbox{7cm}{%
      \begin{itemize}
        \item \textbf{strukturierende} Elemente
        \begin{itemize}
          \item Absätze
          \item Überschriften
        \end{itemize}
        \item \textbf{textflussunterbrechende} Elemente
        \begin{itemize}
          \item Seitenzahlen
          \item Kolumnentitel
          \item Abbildungsunterschriften
          \item Marginalien etc.
        \end{itemize}
      \end{itemize}
    }}
    &
    \raisebox{-\height}{\epsfig{file=figures/layout_rec_example.eps,width=0.4\textwidth}}%
  \end{tabular}
\end{bbawslide}

\begin{bbawslide}{Komponenten eines OCR-Workflows: Layoutanalyse}
  \vspace*{2mm}%
  \centerslidesfalse%
  \begin{tabular}{cc}
    \raisebox{-\height}{\parbox{7cm}{%
      \begin{itemize}
        \item \textbf{strukturierende} Elemente
        \begin{itemize}
          \item Absätze
          \item Überschriften
        \end{itemize}
        \item \textbf{textflussunterbrechende} Elemente
        \begin{itemize}
          \item Seitenzahlen
          \item Kolumnentitel
          \item Abbildungsunterschriften
          \item Marginalien etc.
        \end{itemize}
        \item \textbf{nichttextuelle} Elemente
        \begin{itemize}
          \item Abbildungen
          \item Tabellen
        \end{itemize}
      \end{itemize}
    }}
    &
    \raisebox{-\height}{\epsfig{file=figures/layout_rec_example.eps,width=0.4\textwidth}}%
  \end{tabular}
\end{bbawslide}

\begin{bbawslide}{Komponenten eines OCR-Workflows: Layoutanalyse}
  \vspace*{2mm}%
  \centerslidestrue%
  \textbf{Werkzeuge:}
  \begin{itemize}
    \item auch bei OLR \textbf{Missverhältnis} zwischen Forschungsergebnissen und verfügbaren Lösungen
    \item OCR-Programme implementieren einfache Lösungen zur Page Segmentation, teilweise separat adressierbar
    \begin{itemize} \small
        \item Klassifizierung beschränkt sich im Wesentlichen auf Text vs. Nichttext
        \item Qualität auf schwierigen Vorlagen überschaubar
    \end{itemize}
    \item wissenschaftliche Wettbewerbe und Untersuchungen befassen sich mit der Erkennung \textbf{komplexer Layouts} und \textbf{automatischim Ergebnis nur Einzelbilder auf Zeilenebeneer Dokumentstukturierung}
    \item jedoch \textbf{keine umfassenden} Ansätzen (Kustoden, Marginalien, Bogensignaturen)
    \item ebenfalls kaum Forschung zu \textbf{polygonen Zonen}
  \end{itemize}
\end{bbawslide}

\begin{bbawslide}{Komponenten eines OCR-Workflows: Layoutanalyse}
  \vspace*{2mm}%
  \centerslidestrue%
  \textbf{Werkzeuge:}
  \begin{itemize}
    \item einzelner Befehl für Seiten- und Zeilensegmentierung in \texttt{OCRopus} 
    \begin{itemize}\small
      \item im Ergebnis nur Einzelbilder auf Zeilenebene
      \item \textbf{keine Koordinaten}, kein Zugriff auf Seitensegmentierung
    \end{itemize}
    \item Zugriff auf alle Ebenen der Seitensegmentierung in \texttt{Tesseract}
    \begin{itemize}\small
      \item \textbf{inklusive Koordinaten}
      \item basale Klassifizierung der Segmente (Spalten, Abbildungen, Formeln, Tabellen, Text)
    \end{itemize}
    \item Layouterkennungswerkzeug \texttt{Larex} \hlcite{Reul at al. 2017}
    \begin{itemize}\small
      \item Festlegung buchspezifischer Parameter durch den Nutzer (Spalten, Kolumnentitel etc.) für Klassifizierungsaufgabe
      \item manuelle Nachkorrektur über Benutzeroberfläche
      \item inklusive Zeilenerkennung, open-source: \url{https://github.com/chreul/LAREX}, keine API
    \end{itemize}
  \end{itemize}
\end{bbawslide}

\begin{bbawslide}{Komponenten eines OCR-Workflows: Texterkennung}
  \vspace*{7mm}%
  \centerslidestrue%
  \begin{itemize}
    \item Kernkomponente der OCR
    \item Genauigkeit beeinflusst vom Typ des zugrundeliegenden \textbf{Algorithmus} und vom eingesetzten \textbf{Modell}
    \item aktuell Paradigmenwechsel: \textbf{zeichenorientiert} $\rightarrow$ \textbf{zeilenorientiert}
    \begin{itemize} \small
      \item \textbf{Deep learning:} Tiefe (i.e.~vielschichtige) neuronale Netzwerke zur Sequenzklassifizierung \hlcite{Hochreiter und Schmidhuber 1997}
      \item wesentlich weniger anfällig für \textbf{Zeichenvarianz}
      \item eingebautes \textbf{Sprachmodell}
    \end{itemize}
    \item auch schwierige historische Vorlagen in \enquote{OCR-Reichweite} \hlcite{Springmann 2016}
  \end{itemize}
\end{bbawslide}

\begin{bbawslide}{Komponenten eines OCR-Workflows: Texterkennung}
  \vspace*{7mm}%
  \centerslidestrue%
  \textbf{Werkzeuge:}
  \begin{itemize}
    \item überraschend viele verfügbare OCR-Engines (cf. \url{https://github.com/kba/awesome-ocr})
    \item zwei Platzhirsche im Open-Source-Bereich
    \begin{itemize}
      \item \texttt{Tesseract}
    \end{itemize}
  \end{itemize}
\end{bbawslide}

\begin{bbawpart}{\Large\bf Modelltraining}
\end{bbawpart}

\begin{bbawslide}{Modelltraining}
  \vspace*{7mm}%
  \centerslidestrue%
  \begin{itemize}
    \item Texterkennung auf Basis \textbf{statistischer} Modelle
    \begin{itemize}
      \item Induktion anhand manuell erstellter Trainingsdaten (i.e.~\textbf{Ground Truth})
      \item Wahrscheinlichkeitsverteilung abhängig vom Modelltyp entweder direkt berechnet oder (iterativ) optimiert
    \end{itemize}
    \item unterschiedliche Anforderung an \textbf{Annotationstiefe} abhängig von der OCR-Engine
    \item Qualität und Quantität der Trainingsdaten bestimmt Qualität der Modelle
    \item Kompromiss zwischen \emph{Übertragbarkeit} und spezifischer Textqualität
    \item mitgelieferte Modelle häufig zu \enquote{allgemein}
  \end{itemize}
\end{bbawslide}

\begin{bbawslide}{Modelltraining: Trainingsdaten}
  \vspace*{7mm}%
  \centerslidestrue%
  \begin{itemize}
    \item
  \end{itemize}
\end{bbawslide}

\begin{bbawslide}{Modelltraining: Trainingseffekte}
  \vspace*{7mm}%
  \centerslidestrue%
  \begin{itemize}
    \item
  \end{itemize}
\end{bbawslide}

\begin{bbawpart}{\Large\bf Optimierungsoptionen}
\end{bbawpart}

\begin{bbawslide}{Optimierungsoptionen}
  \vspace*{7mm}%
  \centerslidestrue%
  \begin{itemize}
    \item
  \end{itemize}
\end{bbawslide}

\begin{bbawslide}{Optimierungsoptionen: Lokale Bildoptimierung}
  \vspace*{7mm}%
  \centerslidestrue%
  \begin{itemize}
    \item
  \end{itemize}
\end{bbawslide}

\begin{bbawslide}{Optimierungsoptionen: OCR-Merging}
  \vspace*{7mm}%
  \centerslidestrue%
  \begin{itemize}
    \item
  \end{itemize}
\end{bbawslide}

\begin{bbawslide}{Optimierungsoptionen: OCR-Nachkorrektur}
  \vspace*{7mm}%
  \centerslidestrue%
  \begin{itemize}
    \item
  \end{itemize}
\end{bbawslide}

\begin{bbawslide}{Optimierungsoptionen: Synthetisches Trainingsmaterial}
  \vspace*{7mm}%
  \centerslidestrue%
  \begin{itemize}
    \item
  \end{itemize}
\end{bbawslide}

\begin{bbawpart}{\Large\bf OCR-Workflows}
\end{bbawpart}

\begin{bbawslide}{OCR-Workflows}
  \vspace*{7mm}%
  \centerslidestrue%
  \begin{itemize}
    \item Zugriff auf einzelne Module, Kombination in spezifischem Workflow
  \end{itemize}
\end{bbawslide}

\begin{bbawpart}{\Large\bf Warum braucht man OCR?}
\end{bbawpart}

\begin{bbawslide}{Warum braucht man OCR?}
  \vspace*{7mm}%
  \centerslidestrue%
  \begin{itemize}
    \item
  \end{itemize}
\end{bbawslide}

\begin{bbawpart}{\Large\bf OCR-D}
\end{bbawpart}

\begin{bbawslide}{OCR-D: Überblick}
  \vspace*{7mm}%
  \centerslidestrue%
  \begin{itemize}
    \item DFG-Initiative zur Verbesserung von OCR-Methoden für historische Drucke insbesondere
          für die Volltextdigitalisierung aller in den \emph{Verzeichnissen der im deutschen
          Sprachraum erschienen Drucke} (VD16, VD17, VD18) nachgewiesenen Exemplare
    \item Koordinationsprojekt an Herzog-August Bibliothek Wolfenbüttel, Staatsbibliothek
          Berlin, Karlsruher Institut für Technologie und BBAW $\rightarrow$ Implementierung
          einer Ausschreibung für methodische
          Projekte auf allen Ebenen eines optimierten OCR-Workflows
    \begin{itemize}\small
      \item \ldots
    \end{itemize}
  \end{itemize}
\end{bbawslide}

\begin{bbawslide}{OCR-D: Workflow}
\end{bbawslide}

\begin{bbawslide}{OCR-D: Projektprämissen}
  \vspace*{7mm}%
  \centerslidestrue%
  \begin{itemize}
    \item Lückenschluss zwischen Forschung und Praxis
    \begin{itemize}\small
      \item \ldots
    \end{itemize}
    \item Methodenpluralismus
    \begin{itemize}\small
      \item \ldots
    \end{itemize}
    \item konsequent Open-Source
    \begin{itemize}\small
      \item \ldots
    \end{itemize}
  \end{itemize}
\end{bbawslide}

\begin{bbawslide}{OCR-D: Open-Source-Paradigma}
  \vspace*{7mm}%
  \centerslidestrue%
  \begin{itemize}
    \item öffentlich geförderte Projekte $\leadsto$ öffentlich verfügbare Projektergebnisse
    \item Fehlermeldung und Funktionalitätsfeedback während der Entwicklung
    \item Weiterentwicklung und Pflege auch nach Ablauf der Förderung
    \item Gamera vs. Tesseract
  \end{itemize}
\end{bbawslide}

\begin{bbawslide}{OCR-D: Open-Source-Paradigma}
  \vspace*{2mm}%
  \begin{center}
      \epsfig{file=figures/tesseract_screen.eps,width=0.9\textwidth}%
  \end{center}
\end{bbawslide}

\begin{bbawslide}{OCR-D: Open-Source-Paradigma}
  \vspace*{7mm}%
  \centerslidestrue%
  \begin{itemize}
    \item öffentlich geförderte Projekte $\leadsto$ öffentlich verfügbare Projektergebnisse
    \item Fehlermeldung und Funktionalitätsfeedback während der Entwicklung
    \item Weiterentwicklung und Pflege auch nach Ablauf der Förderung
  \end{itemize}
\end{bbawslide}

\begin{bbawpart}{\Large\bf Danke für Ihre Aufmerksamkeit!\\}
OCR-D-Team: Elisa Hermann, Maria Federbusch, Clemens Neudecker, Ajinkya Prabhune, Matthias Boenig
Mehr zu OCR: \url{https://www.zotero.org/groups/ocr-d}
\end{bbawpart}

\end{document}

%
% modelines
%

%%% Local Variables:
%%% mode: LaTeX
%%% coding: utf-8
%%% tab-width: 2
%%% indent-tabs-mode: nil
%%% End:

% vim: set ts=2 sw=2 expandtab :
